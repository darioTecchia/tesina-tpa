\chapter{Esempio di appplicazione}

  Il nostro esempio è molto semplice, abbiamo una piccola interfaccia grafica
  con un campo di testo dove andremo ad indicare un luogo e, premendo il tasto
  cerca, verrà eseguita una richiesta HTTP di tipo GET alle API di \textit{openweather.org}
  per ottenere la temperatura massima, minima e attuale del luogo indicato.
  
  In questo caso le temperature ci verranno fornite in Kelvin, ma a noi importa
  averle in Celsius.
  
  \section{Metodo tradizionale}
  
  In questo esempio la classe che deve essere adattata è \textbf{\textit{Weather}} che ha il
  metodo 
  \textbf{\textit{get-weather}}. Il Client, nel nostro caso, 
  si aspetta il metodo \textbf{\textit{get-meteo}}.
  
  Per "adattare" \textbf{\textit{Weather}} abbiamo bisogno di una classe adapter,
  \textbf{\textit{MeteoAdapter}}, "l'adattamento", in questo caso, 
  avviene utilizzando il metodo \textbf{\textit{get-weather}} all'interno di 
  \textbf{\textit{get-meteo}}.
  
  Nella seguente pagina andremo ad illustrare il codice python.
  \newpage
  
  \subsection{Codice}
  
  Vediamo il codice dell'esempio:
  
  \textbf{\textit{weather.py}}
  \lstinputlisting[language=Python, frame=single]{code/meteo/weather.py}
  
  \textbf{\textit{meteo.py}}
  \lstinputlisting[language=Python, frame=single]{code/meteo/meteo.py}
  
  \textbf{\textit{main.py}}
  \lstinputlisting[language=Python, frame=single, firstline=2, lastline=7]{code/meteo/main.py}
  
  Come possiamo notare da questo semplice esempio, possiamo identificare
  \textbf{\textit{Adapter}} con \textbf{\textit{MeteoAdapter}}, 
  \textbf{\textit{Adaptee}} con \textbf{\textit{Weather}},
  \textbf{\textit{Request()}} con \textbf{\textit{get-meteo()}} e 
  \textbf{\textit{SpecificRequest()}} con \textbf{\textit{get-weather()}}.
  
  \section{Metodo "Pythonic way"}
  
  Proponiamo anche una variante del primo esempio cercando di sfruttare le potenzialità
  di Python, andando a sostituire il metodo dell'oggetto "legacy" con il 
  metodo che noi ci aspettiamo. 
  
  Come possiamo fare ciò? In Python, ogni oggetto può essere visto come un 
  dizionario accedendo alla proprietà \textbf{\_\_dict\_\_}.
  
  \subsection{Codice}
  
  Vediamo il codice dell'esempio:
  
  \textbf{\textit{adapter.py}}
  \lstinputlisting[language=Python, frame=single]{code/meteo-dict/adapter.py}
  
  \textbf{\textit{weather.py}}
  \lstinputlisting[language=Python, frame=single]{code/meteo-dict/weather.py}
  
  \textbf{\textit{main.py}}
  \lstinputlisting[language=Python, frame=single, firstline=5, lastline=10]{code/meteo-dict/main.py}
  
  